\begin{frame}
  \begin{center}
    \LARGE <<Латинский квадрат>>
  \end{center}
  \begin{center}
      \includegraphics[width=5cm]{memes/h-meme.jpg}
  \end{center}
  \begin{itemize}
  \item Идея задачи --- Михаил Тихомиров, МФТИ
  \item Разработка задачи --- Николай Будин, ИТМО
  \end{itemize}
\end{frame}

\begin{frame}{Постановка задачи}

  \begin{itemize}
  \item Дана матрица $n \times m$
  \item Нужно найти количество подматриц, которые являются латинскими квадратами
  \item Латинский квадрат~--- матрица $k \times k$, в которой ровно $k$ различных элементов и в каждой строке и каждом столбце элементы не повторяются
  \end{itemize}
  
\end{frame}

\begin{frame}{Решение за $\O(n ^ 5)$ (9 баллов)}
  \begin{itemize}
  \item Переберем подматрицу, являющуюся квадратом
  \item Проверим, что подматрица является латинским квадратом за её размер
  \item В зависимости от используемых для проверки структур данных, асимптотика может дополнительно умножиться на логарифм
  \end{itemize}
\end{frame}

\begin{frame}{Решение за $\O(n ^ 4)$ (10 баллов)}
  \begin{itemize}
  \item Квадратная подматрица является латинским квадратом, если:
  \begin{itemize}
  \item В каждой строке и каждом стролбце все элементы различны
  \item Количество различных элементов в подматрице равно длине её стороны
  \end{itemize}
  \end{itemize}
\end{frame}

\begin{frame}{Решение за $\O(n ^ 4)$ (10 баллов)}
  \begin{itemize}
  \item Переберём верхний левый угол латинского квадрата
  \item Будем перебирать длину стороны квадрата в порядке возрастания
  \item При увеличении стороны на $1$, для всех добавившихся элементов проверим, что не появилось дубликатов в какой-то строке или столбце
  \item Для этого будем для каждой строки и столбца хранить для каждого символа, встречается ли он там
  \end{itemize}
\end{frame}

\begin{frame}{Решение за $\O(n ^ 4)$ (10 баллов)}
  \begin{itemize}
  \item Будем поддерживать количество различных элементов в текущем квадрате
  \item Для этого нужно увеличивать счётчик на $1$, когда мы впервые встретили какой-то элемент
  \item Для фиксированного верхнего левого угла решение работает за $\O(n \cdot m)$
  \end{itemize}
\end{frame}

\begin{frame}{Решение за $\O(n ^ 3 \cdot \log(n)$ (25 баллов)}
  \begin{itemize}
  \item Для каждой клетки найдём ближайшую снизу и ближайшую справа клетки с таким же значением
  \item Запишем пару чисел: номер строки для клетки снизу и номер столбца для клетки справа
  \item На получившейся матрице из пар чисел построим разреженную таблицу
  \item Операция~--- поэлементный минимум
  \end{itemize}
\end{frame}

\begin{frame}{Решение за $\O(n ^ 3 \cdot \log(n)$ (25 баллов)}
  \begin{itemize}
  \item Теперь для квадрата можно за $\O(1)$ проверить, встречаются ли в какой-то строке или столбце дубликаты
  \item Пусть мы проверяем квадрат с верхним левым углом в $(x, y)$ и стороной $s$
  \item Сделаем запрос к разреженной таблице и проверим, что первый элемент не меньше $x + s$, а второй~--- не меньше $y + s$
  \end{itemize}
\end{frame}

\begin{frame}{Решение за $\O(n ^ 3 \cdot \log(n)$ (25 баллов)}
  \begin{itemize}
  \item Осталось проверить, что в квадрате ровно $s$ различных чисел
  \item Проверку можно сделать с помощью хешей
  \item Каждому значению назначим случайное число от $0$ до $2^{64} - 1$
  \end{itemize}
\end{frame}

\begin{frame}{Решение за $\O(n ^ 3 \cdot \log(n)$ (25 баллов)}
  \begin{itemize}
  \item Вычислим сумму в квадрате и сравним с суммой в первой строке квадрата, умноженной на $s$
  \item Если не равны, квадрат точно не латинский
  \item Если равны, то будем считать, что латинский
  \item Вероятность ошибиться мала
  \item Для вычисления сумм на прямоугольниках, можно посчитать префиксные суммы
  \item Суммы вычисляем по модулю $2^{64}$
  \end{itemize}
\end{frame}


\begin{frame}{Полное решение за $\O(n ^ 2 \cdot \log^2(n))$}
  \begin{itemize}
  \item Пусть есть латинский квадрат $k \times k$
  \item Рассмотрим его подквадрат $l \times l$
  \item Если $l > \frac{k}{2}$, то подквадрат содержит все $k$ различных значений, которые встречаются в исходном квадрате
  \item Следствие: если один латинский квадрат вложен в другой, их размеры отличаются хотя бы в $2$ раза
  \end{itemize}
\end{frame}

\begin{frame}{Полное решение за $\O(n ^ 2 \cdot \log^2(n))$}
  \begin{itemize}
  \item Для фиксированного верхнего левого угла есть максимум один латинский квадрат со стороной, лежащей в полуинтервале $[2^q, 2^{q + 1})$ для любого целого $x$
  \item Чтобы узнать сторону этого квадрата можно узнать количество различных значений внутри квадрата со стороной $2^q$
  \item После этого можно за $\O(1)$ проверить квадрат с получившейся стороной
  \end{itemize}
\end{frame}

\begin{frame}{Полное решение за $\O(n ^ 2 \cdot \log^2(n))$}
  \begin{itemize}
  \item Осталось для всех возможных позиций квадрата со стороной $2^q$ вычислить количество различных значений в нём
  \item Для этого будем перебирать полосу из $2^q$ строк сверху вниз
  \item Для каждого значения будем поддерживать множество столбцов на которых оно встречается в текущей полосе строк
  \end{itemize}
\end{frame}

\begin{frame}{Полное решение за $\O(n ^ 2 \cdot \log^2(n))$}
  \begin{itemize}
  \item Будем поддерживать массив $a_y$~--- количество различных элементов в квадрате с левой границей $y$
  \item Когда множество столбцов для некоторого значения изменяется, на некотором отрезке массива $a$ нужно прибавить или вычесть $1$
  \end{itemize}
\end{frame}

\begin{frame}{Полное решение за $\O(n ^ 2 \cdot \log^2(n))$}
  \begin{itemize}
  \item В алгоритме $\log(n)$ итераций по $q$
  \item Каждая итерация работает за $\O(n \cdot m \cdot \log(n))$
  \end{itemize}
\end{frame}

\begin{frame}
  \begin{center}
    \LARGE <<Лапша быстрого приготовления>>
  \end{center}

  \begin{center}
      \includegraphics[width=5cm]{memes/f-meme.jpg}
  \end{center}

  \begin{itemize}
  \item Идея задачи --- Григорий Резников, МГУ
  \item Разработка задачи --- Егор Чунаев, Яндекс
  \end{itemize}

\end{frame}

\begin{frame}{Постановка задачи}

  \begin{itemize}
  \item Дан двудольный граф
  \item $f(S)$ --- сумма чисел в соседях множества вершин левой доли
  \item Необходимо найти gcd всех чисел $f(S)$
  \end{itemize}
  
\end{frame}

\begin{frame}{Решение за $\O(2^n\cdot m)$ (21 баллов)}
  \begin{itemize}
  \item Переберем все множества вершин левой доли, и посчитаем $f(S)$
  \item Необходима аккрутная реализация
  \item В случае проблем с time limit, например можно использовать битовые маски
        и предподсчитать суммы во всех подмножествах в правой доле
  \end{itemize}
\end{frame}

\begin{frame}{Полное решение}
  \begin{itemize}
  \item Если у вершины нет соседей в левой доле, ее можно удалить
  \item Если у вершин одинаковое множество соседей в левой доле, их можно заменить на вершину с суммой
  \item Ответ --- gcd чисел в оставшихся вершинах. В зависимости от эффективности реализации может получить 54 балла
  \end{itemize}
\end{frame}

\begin{frame}{Полное решение. Доказательство}
  \begin{itemize}
  \item Ответ делится на этот gcd. Поделим все числа на него, и докажем, что ответ 1.
  \item Выберем произвольное $k$, и докажем, что есть $S$, такое что $f(S)$ не делится на $k$
  \item Если сумма всех чисел делится на $k$, то вся левая доля подходит
  \end{itemize}
\end{frame}

\begin{frame}{Полное решение. Доказательство}
  \begin{itemize}
  \item Выберем вершину $v$ которая не делится на $k$ с самым маленьким количеством соседей
  \item Пусть $S$ множество вершин не связных с ней
  \item $f(S) = \sum\limits_{u}{c_u} - \sum\limits_{u | N(u) \subset N(v)}{c_u} - c_v$
  \item Первое слагаемое делится на $k$, иначе уже нашли ответ. Второе делится на $k$, так как у $v$ минимальное количетсво соседей. 
        Последнее не делится на $k$, значит и сумма не делится на $k$.
  \end{itemize}
\end{frame}

\begin{frame}
  \begin{center}
    \LARGE <<Лапша быстрого приготовления>>
  \end{center}

  \begin{center}
      \includegraphics[width=5cm]{memes/f-meme.jpg}
  \end{center}

  \begin{itemize}
  \item Идея задачи --- Григорий Резников, МГУ
  \item Разработка задачи --- Егор Чунаев, Яндекс
  \end{itemize}

\end{frame}

\begin{frame}{Постановка задачи}

  \begin{itemize}
  \item Дан двудольный граф
  \item $f(S)$ --- сумма чисел в соседях множества вершин левой доли
  \item Необходимо найти gcd всех чисел $f(S)$
  \end{itemize}
  
\end{frame}

\begin{frame}{Решение за $\O(2^n\cdot m)$ или $\O(2^n \cdot n)$ (21~балл)}
  \begin{itemize}
  \item Переберем все множества вершин левой доли, и посчитаем $f(S)$
  \item Необходима аккуратная реализация
  \item Чтобы получить константно более эффективное решение можно сохранить граф в битовых масках, и заранее
    для каждого подмножества вершин справа предподсчитать сумму по все подмножествам.
  \end{itemize}
\end{frame}

\begin{frame}{Полное решение}
  \begin{itemize}
  \item Если у вершины нет соседей в левой доле, ее можно удалить
  \item Если у вершин одинаковое множество соседей в левой доле, их можно заменить на вершину с суммой
  \item Ответ --- gcd чисел в оставшихся вершинах. В зависимости от эффективности реализации получится 54 или 100 баллов.
  \end{itemize}
\end{frame}

\begin{frame}{Полное решение. Доказательство}
  \begin{itemize}
  \item Ответ делится на этот gcd. Поделим все числа в вершинах справа на него, и докажем, что для этой задачи ответ будет 1.
  \item Выберем произвольное $k$, и докажем, что есть $S$, такое что $f(S)$ не делится на $k$
  \item Если сумма всех чисел не делится на $k$, то вся левая доля подходит.
  \end{itemize}
\end{frame}

\begin{frame}{Полное решение. Доказательство}
  \vspace{-0.5em}
  \begin{itemize}
  \item Выберем правую вершину $t$ которая по весу не делится на $k$ и с самым маленьким количеством соседей.
  \item Пусть $S$ множество левых вершин не смежных с ней.
  \item $f(S) = \sum\limits_{u \in \t{all}}{c_u}-\sum\limits_{\substack{u, \text{~что~} \\ N(u) \subset N(t)}}{c_u} ~- ~~~~c_t$
  \item Первое слагаемое делится на $k$, иначе уже нашли ответ. Второе делится на $k$, так как у $v$ минимальное количество соседей (у $u$ получается строго меньше).
        Последнее слагаемое не делится на $k$, значит и сумма не делится на $k$.
  \end{itemize}
\end{frame}
